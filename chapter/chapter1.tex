\chapter{Einleitung}
\label{chap:einleitung}
\erstelltvon{Treulieb}
Schwerpunkt der Arbeit wird die Vorstellung und Erklärung von drei verschiedenen Algorithmen zum
Zeichnen von Bäumen im Ebenen-Layout sein. Dabei wird das Hauptaugenmerk auf dem Zeichnen
von Binärbäumen liegen. Hierfür wurden drei Algorithmen betrachtet, ein naiver und ein verbesserter von
Wetherhell und Shannon, und einer von Tilford und Reingold.

\section{Motivation}
\label{sec:motivation}
Eine ästhetisch ansprechende Darstellung von Bäumen ist nicht trivial.
Viele Algorithmen positionieren die Knoten derart, sodass die Übersichtlichkeit
und Verständlichkeit hinter der Abbildung verloren gehen.

\section{Zielsetzung}
\label{sec:zielsetzung}
\erstelltvon{Treulieb}
Ziel dieser Arbeit ist, ein Verständnis dafür zu schaffen, wie diese Algorithmen funktionieren
und was das Ergebnis für einen bestimmten (Binären)-Baum ist. Ferner sollen die
Algorithmen in Java implementiert werden, um eigene Bäume zeichnen zu können.
