\chapter{Einleitung}
\label{chap:einleitung}
\erstelltvon{Garan/Treulieb}
Schwerpunkt der Arbeit wird die Vorstellung und Erklärung von drei verschiedenen Algorithmen zum
Zeichnen von Bäumen im Ebenen-Layout sein. Dabei wird das Hauptaugenmerk auf dem Zeichnen
von Binärbäumen liegen. Hierfür wurden drei Algorithmen betrachtet, ein naiver und ein verbesserter Algorithmus von
Wetherhell und Shannon sowie einer von Reingold und Tilford. 

\section{Motivation}
\label{sec:motivation}
Das Verwenden von Bäumen als Datenstruktur bietet eine Möglichkeit zur komprimierten und sortierten Darstellung von Daten, 
sowohl intern im Programmcode als auch extern als Modell. Wetherell und Shannon erkennen dies in ihrer Arbeit und schrieben 
"[...] a good drawing of a tree is often a powerful intuitive guide to a modeled problem [...]" \cite[]{q1}. In der Praxis können Bäume 
zum Beispiel zur Sortierung bzw. Speicherung von Daten oder zur Darstellung von unternehmensinternen Hierarchien verwendet werden. 

\section{Zielsetzung}
\label{sec:zielsetzung}
Ziel dieser Arbeit ist, ein Verständnis dafür zu schaffen, wie diese Algorithmen funktionieren
und was das Ergebnis für einen bestimmten (binären) Baum ist. Ferner sollen die
Algorithmen in Java implementiert werden, um eigene Bäume zeichnen zu können. Dabei werden auch die Vor- und Nachteile der einzelnen Algorithmen
betrachtet. Zudem sollen anhand der Implementierungen in Java
eine weitere Möglichkeit zur Implementierung der Algorithmen gezeigt werden. 
