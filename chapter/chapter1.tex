\chapter{Einleitung}
\label{chap:einleitung}

Schwerpunkt dieses Praxisberichts wird die Analyse der vorhandenen Daten über Projekte der Siemens Mobility GmbH sein, welche Anwendungsregeln bewertet haben. Zu Beginn wird zunächst definiert, was
Anwendungsregeln und Anforderungen sind und was das Anforderungsmanagement-Tool IBM Rational Doors ist. Danach werden zwei Phasen eines Data-Mining Modells auf den vorhandenen Daten angewandt. Zum Schluss
soll eine .csv-Datei erstellt werden, welche einen Datensatz mit allen bewerteten Anwendungsregeln beinhalten soll. 

\section{Motivation}
Der Grund für die Analyse und Aufbereitung der Daten zu bewerteten Anwendungsregeln ist das anschließende Anlerenen eines unterstützenden KI-Systems. Jenes KI-System soll in der Lage sein,
Requirements Managern beim Bewerten von Anwendungsregeln zu unterstützen, indem es dem Requirements Manager Vorschläge zur Bewertung jeder Anwendungsregeln basierend auf vorherigen Bewertungen zu
eben jener Anwendungsregel gibt.

\section{Zielsetzung}
\label{chap:Zielsetzung}
Ziel des Praxisprojektes ist es, dass aus der Menge an Daten von bereits abgeschlossenen und noch laufenden Projekten der Siemens Mobility GmbH, die Daten herausgefiltert werden, die 
zum Anlernen des KI-Systems geeignet sind. Zudem sollen die Daten auch so aufbereitet werden, dass ausschließlich für das Anlernen relevante Daten vorhanden sind. Mithilfe dieser Daten
soll dann das KI-System angelernt werden, was jedoch nicht mehr Teil dieser Arbeit ist, sondern im Anschluss in einer Bachelorarbeit stattfinden soll, welche auf dieses Praxisprojekt und den 
dazugehörigen Praxisbericht aufbauen soll.

