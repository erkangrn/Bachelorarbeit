\chapter{Einleitung}
\label{chap:einleitung}
\ac{KI} ist ein Thema von zunehmender Bedeutung, das sowohl im alltäglichen Leben als auch im industriellen Umfeld immer mehr Anwendung und Bedeutung findet. Das Potenzial von \ac{KI}
reicht dabei von der Unterstützung bis hin zur kompletten Automatisierung von Aufgaben und Tätigkeiten. Dieses Potenzial soll bei der Bewertung von Anwendungsregeln genutzt werden.
Anwendungsregeln sind spezielle Anforderungenen, die Projekte beachten und einhalten müssen, um eine sichere Nutzung zu ermöglichen.
Das Bewerten dieser Anwendungsregeln ist dabei Aufgabe der Requirements-Manager.  

\section{Motivation}
\label{chap:Motivation}
Nachdem ein zuständiger Requirements-Manager für sein Projekt die Anwendungsregeln der Komponenten, die von dem Projekt genutzt werden, importiert hat, muss er im Anschluss daran
diese Anwendungsregeln bewerten. Die importierten Anwendungsregeln wurden dabei wahrscheinlich in der Vergangenheit bereits von anderen Projekten bewertet. 
Es ist für einen Requirements-Manager nicht möglich, alle Bewertungen zu sämtlichen Anwendungsregeln zu kennen. Zudem können die Bewertungen von anderen Projekten mehr oder weniger relevant 
sein für das aktuelle Projekt. Deshalb soll ein Modell entworfen werden, das Requirements-Managern bei der Bewertung von Anwendungsregeln, mithilfe von künstlicher Intelligenz, unterstützen soll.

\section{Zielsetzung}
\label{chap:Zielsetzung}
Ziel dieser Bachelorarbeit ist das Entwerfen eines Modells für ein \ac{KI}-System, das in der Lage sein soll, konkrete Vorschläge zur Bewertung von Anwendungsregeln zu treffen.
Ein solcher Vorschlag soll anhand der Bewertungen in der Vergangenheit generiert werden und sowohl den Status der Bewertung als auch ein Statement dazu beinhalten.
Dabei soll dieses Modell ausdrücklich nicht die Aufgabe komplett übernehmen, sondern lediglich als unterstützendes Tool dienen. 
Um dieses Ziel zu erreichen, muss dazu ein Datensatz erstellt werden, mit dem dieses Modell anschließend trainiert werden kann. Die dafür benötigten Schritte zur 
Datenverarbeitung werden in dieser Arbeit ebenfalls vorgestellt. 

\section{Aufbau}
Zu Beginn der Arbeit wird definiert, was \ac{RE} ist, was Anforderungen und Anwendungsregeln genau sind und wofür sie benötigt werden. Zudem wird vorgestellt,
wie Anwendungsregeln bei der Siemens Mobility GmbH bewertet werden. Anschließend wird erläutert, was \ac{KI} ist und worin sich \ac{KI} vom maschinellen Lernen sowie dem \ac{DL}
unterscheidet. Es wird gezeigt, wie eine \ac{KI} als neuronales Netz aufgebaut wird und wie ein solches neuronales Netz trainiert werden kann. 
Nachdem die beiden Themengebiete \ac{RE} und \ac{KI} grundlegend behandelt wurden, werden die in dieser Arbeit verwendeten Tools und Programmiersprachen,
zusammen mit den in der Arbeit genutzten Programmierbibliotheken, vorgestellt. Danach wird der Datensatz, der im Praxisprojekt erstellt wurde, importiert
und mit weiteren Daten erweitert. Dieser neue Datensatz wird daraufhin in eine Form gebracht, der für das Anlernen eines \ac{KI}-Modells geeignet ist.
Wenn dieser Schritt erledigt ist, kann das \ac{KI}-Modell erstellt werden. Dabei wird gezeigt, wie solch ein Modell definiert werden kann und welche 
Verbesserungsmöglichkeiten existieren. Zum Schluss wird mit diesem neu erstellten Modell der erste Vorschlag zur Vorhersage einer Bewertung einer Anwendungsregel getroffen.
