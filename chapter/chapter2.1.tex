\section{Requirements Engineering}
\label{chap:Requirements Engineering}
Nach der Definition des International Requirements Engineering Boards(IREB) bezeichnet das Requirements Engineering die systematische 
und disziplinierte Vorgehensweise bei der Spezifikation und dem Management von Anforderungen. Das Ziel des Requirements Engineering ist dabei, 
die Wünsche und Bedürfnisse der Stakeholder zu verstehen \cite[vgl. S.30]{IREB_Glossary}. Stakeholder sind Personen oder Organisationen, die die 
Anforderungen des Systems direkt oder indirekt beeinflussen oder die von dem System betroffen sind \cite[vgl. S.33]{IREB_Glossary}. 
Beispielsweise können Kunden oder Nutzer, aber auch der Gesetzgeber, potentielle Stakeholder sein. Außerdem soll das Risiko minimiert werden, 
dass diese Wünsche und Bedürfnisse nicht oder nur unzureichend erfüllt werden \cite[vgl. S.30]{IREB_Glossary}.

Einen Teilbereich des Requirements Engineering stellt das \ac{RM} dar. Dieser Prozess beschreibt die Verwaltung, Speicherung, 
Änderung sowie die Rückverfolgung von Anforderungen \cite[vgl. vgl. S.8]{IREB_Glossary}.

\subsection{Anforderungen}
Die IEEE definiert eine Anforderung wie folgt: 
\begin{quote}
    (1) A condition or capability needed by a user to solve a problem or achieve an objective.\\
    (2) A condition or capability that must be met
        or possessed by a system or system component to satisfy a contract, standard, specification, or other formally imposed documents.\\
    (3) A documented representation of a condition or capability as in (1) or (2). \cite[S.62]{IEEE_Glossary}
\end{quote}

Daher bilden Anforderung die Basis eines jeden Projekts, da diese definieren, welche Bedingungen ein System erfüllen muss bzw. welche Fähigkeiten 
es besitzen muss. Sie werden idealerweise unter Berücksichtigung und in Zusammenarbeit mit den Stakeholdern des Projekts ermittelt. Neben den 
Stakeholdern können unter Anderem auch Normen, Gesetze oder Vorgänger eines Systems weitere Quellen für Anforderungen sein. Um von jedem 
Partizipierendem des Projekts verstanden zu werden, werden Anforderungen in der Regel in natürlicher Sprache formuliert. Da natürliche Sprache Raum für 
Interpretation bieten kann, muss darauf geachtet werden, dass die Anforderungen so klar und unmissverständlich wie möglich formuliert werden und dabei 
trotzdem vollständig bleiben. Zudem wird voraussichtlich nicht jeder Stakeholder über die fachlichen Kenntnisse verfügen um Fachsprache oder 
Konventionen zu verstehen, weshalb darauf verzichtet werden sollte \cite[vgl. S.2]{DOORS}. Um sicherzustellen, dass Anforderungen korrekt 
formuliert werden, wurden in der Norm ISO/IEC/IEEE 29148:2018(E) Eigenschaften definiert, welche Anforderungen erfüllen sollen. 
Diese Eigenschaften werden nachfolgend dargestellt und kurz beschrieben:

\begin{enumerate}[leftmargin=*,labelindent=16pt,font=\bfseries, nolistsep]
    \item[Notwendig] Die Anforderung definiert eine wesentliche Fähigkeit, Eigenschaft, Einschränkung und/oder einen Qualitätsfaktor \cite[vgl. S.12]{RE-ISO}.
    \item[Angemessen] Die Anforderung verfügt über einen angemessenen Detaillierungsgrad und erlaubt dabei bei der Implementierung größtmögliche Unabhängigkeit \cite[vgl. S.12]{RE-ISO}.
    \item[Eindeutig] Die Anforderung ist leicht zu verstehen, einfach formuliert und kann nur auf eine einzige Weise interpretiert werden \cite[vgl. S.12]{RE-ISO}.
    \item[Komplett] Die Anforderung ist hinreichend beschrieben und benötigt keine weiteren Informationen um verstanden zu werden \cite[vgl. S.12]{RE-ISO}.
    \item[Atomar] Die Anforderung beschreibt eine einzige Fähigkeit oder Bedingung \cite[vgl. S.12]{RE-ISO}.
    \item[Durchführbar] Die Anforderung kann innerhalb der Beschränkungen des Systems mit akzeptablem Risiko durchgeführt werden \cite[vgl. S.13]{RE-ISO}.
    \item[Verifizierbar] Die Umsetzung der Anforderung kann überprüft werden \cite[vgl. S.13]{RE-ISO}.
    \item[Korrekt] Die Anforderung ist eine genaue Darstellung des Bedürfnisses ihrer Quelle \cite[vgl. S.13]{RE-ISO}.
    \item[Konform] Die Anforderung wurde, wenn möglich, mithilfe einer genehmigten Standardvorlage und -stil verfasst \cite[vgl. S.13]{RE-ISO}.
\end{enumerate}

Nach einer Umfrage aus dem Chaos Report der Standish Group nennen mehr als die Hälfte der Befragten als Faktor für die Beeinträchtigung von Projekten
einen Grund, der direkt im Zusammenhang mit mangelndem \ac{RE} und \ac{RM} steht. Dazu gehören z.B. Gründe wie unvollständige Anforderungen, Nutzer nicht
ausreichend involviert, unrealistische Erwartungen, geänderte Anforderungen und Spezifikationen \cite[vgl. S.5]{Chaos}. Gut durchgeführtes 
\ac{RE} und \ac{RM} ist also essentiell für den Erfolg von Projekten.

\subsection{Bewerten von Anwendungsregeln}
