\section{Requirements Engineering}
\label{chap:Requirements Engineering}
Nach der Definition des International Requirements Engineering Boards(IREB) bezeichnet das Requirements Engineering die systematische und disziplinierte Vorgehensweise bei der Spezifikation
und dem Management von Anforderungen. Das Ziel des Requirements Engineering ist dabei, die Wünsche und Bedürfnisse der Stakeholder zu verstehen \cite[S.30]{IREB_Glossary}. Stakeholder sind Personen 
oder Organisationen, die die Anforderungen des Systems direkt oder indirekt beeinflussen oder die von dem System betroffen sind \cite[S.33]{IREB_Glossary}. Beispielsweise können Kunden oder Nutzer, 
aber auch der Gesetzgeber, potentielle Stakeholder sein. Außerdem soll das Risiko minimiert werden, dass diese Wünsche und Bedürfnisse nicht oder nur unzureichend erfüllt werden \cite[S.30]{IREB_Glossary}.

Einen Teilbereich des Requirements Engineering stellt das Requirements Management dar. Dieser Prozess beschreibt die Verwaltung, Speicherung, Änderung sowie die Rückverfolgung 
von Anforderungen \cite[S.8]{IREB_Glossary}.
\subsection{Anforderungen}
Die IEEE definiert eine Anforderung wie folgt: 
\begin{quote}
    (1) A condition or capability needed by a user to solve a problem or achieve an objective.\\
    (2) A condition or capability that must be met
        or possessed by a system or system component to satisfy a contract, standard, specification, or other formally imposed documents.\\
    (3) A documented representation of a condition or capability as in (1) or (2). \cite[S.62]{IEEE_Glossary}
\end{quote}
Daher bilden Anforderung die Basis eines jeden Projekts, da diese definieren, welche Bedingungen ein System erfüllen muss bzw. welche Fähigkeiten es besitzen muss. Sie werden idealerweise 
unter Berücksichtigung und in Zusammenarbeit mit den Stakeholdern des Projekts ermittelt. Neben den Stakeholdern können unter Anderem auch Normen, Gesetze oder Vorgänger eines Systems 
weitere Quellen für Anforderungen sein.     

\subsection{Bewerten von Anwendungsregeln}
