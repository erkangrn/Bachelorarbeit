\section{Python}
\label{chap:Python}


\subsection{Keras}
Keras ist eine Bibliothek für die Programmiersprache Python, welche Klassen und Funktionen liefert, um verschiedene \ac{DL}-Modelle zu erstellen
und diese anschließend zu trainieren. Operationen zur Berechnung und Bearbeitung von beispielsweise Tensoren bringt diese Bibliothek nicht mit.
Stattdessen greift Keras auf andere Bibliotheken zurück, die diese Funktionalitäten mitbringen, und nutzt diese dann. Dabei ist Keras kompatibel zu mehreren
Bibliotheken dieser Art, wie zum Beispiel TensorFlow von Google oder CNTK von Microsoft. Der Entwickler kann aussuchen, welche dieser Bibliotheken
verwenden will und kann diese auch während der Entwicklung wechseln \cite[vgl. S.89ff.]{DL_PY}. François Chollet empfiehlt standardmäßig TensorFlow zu nutzen, da diese 
\glqq am weitesten verbreitet, skalierbar und ausgereift\grqq{}\cite[S.91]{DL_PY} sei. Was TensorFlow genau kann und wie es genutzt wird, ist für die Erstellung eines
\ac{DL}-Modells mit Keras nicht relevant und wird somit nicht genauer beschrieben.

Das Erstellen eines \ac{NN} mithilfe von Keras folgt dabei in der Regel den folgenden vier Schritten:
\begin{enumerate}
    \item Trainingsdatensatz definieren und in Ein- und Ausgabewerte aufteilen \cite[vgl. S.92.]{DL_PY}
    \item \ac{NN} definieren, indem die einzelnen Schichten konfiguriert werden \cite[vgl. S.92.]{DL_PY}
    \item Verlustfunktion, Optimierer und Metrik(Kennzahl) auswählen \cite[vgl. S.92.]{DL_PY}
    \item Modell mit Trainingsdaten anlernen \cite[vgl. S.92.]{DL_PY}
\end{enumerate}
Ein großer Vorteil der Bibliothek ist, dass Keras bereits die verschiedenen Schichten als Klassen mitbringt, diese also nicht vom Entwickler erst definiert werden müssen.
Der Entwickler kann dadurch einem Modell beliebig Schichten hinzufügen oder entfernen und diese nach seinen Wünschen konfigurieren. 
Die Anzahl an Neuronen innerhalb einer Schicht und die Aktivierungsfunktion der Neuronen in der Schicht werden der Schicht einfach als Parameter übergeben 
und die Aktivierungsfunktion muss ebenfalls nicht selbst definiert werden, denn da reicht es aus, den Namen der Funktion als Parameter anzugeben.
Das Auswählen der Verlustfunktion, des Optimierers und der Metrik verlaufen ebenfalls genauso einfach. Diese werden, nachdem das Modell mit seinen Schichten 
definiert wurden, ganz simpel, wie bei der Auswahl der Aktivierungsfunktion, einer Funktion als Parameter übergeben.
Genau diese Schritte werden in dieser Arbeit durchlaufen, um das \ac{KI}-System zur Bewertung von Anwendungsregeln zu erstellen und anzulernen.

\subsection{Pandas}

Um unter Anderem das Importieren der Daten, welche im Praxisprojekt gesammelt wurden, zu ermögliche, wird die Datenverarbeitungsbibliothek Pandas genutzt.
Pandas ist ein Akronym, welches für panel data steht. Diese Bibliothek basiert dabei auf Tabellen, ähnlich wie bei Excel, und bietet die Möglichkeit
Excel-Dateien, CSV-Dateien und weitere Dateitypen direkt zu importieren oder zu exportieren. Die beiden wichtigsten Datenstrukturen sind dabei die
Series und die Dataframe \cite[vgl. S.3]{Pandas}. 

Series... 

Dataframes... 

\subsection{Matplotlib}
Erstellen von Grafiken