\section{Python}
\label{chap:Python}

Python ist eine interpretierte, high-level, objektorientierte Programmiersprache und wird von einem Artikel 
im International Research Journal of Engineering and Technology (IRJET) als die am schnellsten wachsende Programmiersprache 
bezeichnet \cite[vgl. S.354]{Python}. Jacqueline Kazil, ehemaliges Vorstandsmitglied der Python Software Foundation,
nennt als Hauptgründe für das Wachstum der Programmiersprache die Beliebtheit von Python in den Themen \ac{ML} und Data Science \cite[vgl. S.354]{Python}.
Weitere Eigenschaften der Programmiersprache sind:
\begin{itemize}
    \item Hohe Lesbarkeit durch einfache Syntax \cite[vgl. S.354]{Python}
    \item Programme haben weniger Zeilen Code als vergleichbare Sprachen wie C \cite[vgl. S.354]{Python}
    \item hohe Flexibilität durch dynamische Typisierung \cite[vgl. S.354]{Python}
    \item Automatisches Speichermanagement \cite[vgl. S.354]{Python}
    \item läuft auf vielen verschiedenen Betriebssystemen \cite[vgl. S.355]{Python}
\end{itemize}
Die dynamische Typisierung sorgt zwar auf der einen Seite für eine erhöhte Flexibilität, da Variablen kein fester Typ bei der Deklarierung zugeordnet werden muss,
liefert aber auf der anderen Seite auch Nachteile. Durch die dynamische Typisierung kann die Ausführung eines Programms zeitintensiver werden und bei größeren 
Projekten kann es dazu kommen, dass nicht mehr genau nachvollzogen werden kann, welchen Typ eine Variable hat \cite[vgl. S.355]{Python}. Beide Nachteile sind bei der Größe
dieses Projekts aber vernachlässigbar. 

Ein großer Vorteil von Python sind die Programmierbibliotheken, die die Sprache besitzt und die importiert werden können. Für nahezu jeden Anwendungsfall existiert eine 
Bibliothek die Python um weitere Klassen und Funktionen erweitert und somit die Entwicklung von Anwendungen beschleunigt und erleichtert. Durch sie muss 
der Entwickler die Funktionalitäten, die eine Bibliothek mitbringt, nicht selber implementieren und kann stattdessen auf diese Bibliotheken zurückgreifen. 
In dieser Arbeit wurden drei Bibliotheken verwendet, die für das Erstellen von \ac{KI}-Modellen, für die Datenverarbeitung und das Visualisieren von Daten
genutzt wurden. Diese werden in den folgenden Kapiteln näher beschrieben.

Aufgrund der Beliebtheit der Sprache im Thema \ac{KI} und den Programmierbibliotheken wird Python in dieser Arbeit dazu genutzt,
die Daten aus dem Praxisprojekt zu importieren, zu verarbeiten und das \ac{KI}-Modell zu erstellen und mit den Daten anzulernen.
Eine Alternative dazu wäre die Programmiersprache R, jedoch wird hier Python bevorzugt, da Python weiter verbreitet ist und eine simplere Syntax hat und somit 
einfacher zu lesen und zu verstehen ist.

\subsection{Keras}
\label{chap:Keras}
Keras ist eine Bibliothek für die Programmiersprache Python, welche Klassen und Funktionen liefert, um verschiedene \ac{DL}-Modelle zu erstellen
und diese anschließend zu trainieren. Operationen zur Berechnung und Bearbeitung von beispielsweise Tensoren bringt diese Bibliothek nicht mit.
Stattdessen greift Keras auf andere Bibliotheken zurück, die diese Funktionalitäten mitbringen, und nutzt diese dann. Dabei ist Keras kompatibel zu mehreren
Bibliotheken dieser Art, wie zum Beispiel TensorFlow von Google oder CNTK von Microsoft. Der Entwickler kann aussuchen, welche dieser Bibliotheken
verwenden will und kann diese auch während der Entwicklung wechseln \cite[vgl. S.89ff.]{DL_PY}. François Chollet empfiehlt standardmäßig TensorFlow zu nutzen, da diese 
\glqq am weitesten verbreitet, skalierbar und ausgereift\grqq{}\cite[S.91]{DL_PY} sei. Was TensorFlow genau kann und wie es genutzt wird, ist für die Erstellung eines
\ac{DL}-Modells mit Keras nicht relevant und wird somit nicht genauer beschrieben.

Das Erstellen eines \ac{NN} mithilfe von Keras folgt dabei in der Regel den folgenden vier Schritten:
\begin{enumerate}
    \item Trainingsdatensatz definieren und in Ein- und Ausgabewerte aufteilen \cite[vgl. S.92]{DL_PY}
    \item \ac{NN} definieren, indem die einzelnen Schichten konfiguriert werden \cite[vgl. S.92]{DL_PY}
    \item Verlustfunktion, Optimierer und Metrik(Kennzahl) auswählen \cite[vgl. S.92]{DL_PY}
    \item Modell mit Trainingsdaten anlernen \cite[vgl. S.92]{DL_PY}
\end{enumerate}
Ein großer Vorteil der Bibliothek ist, dass Keras bereits die verschiedenen Schichten als Klassen mitbringt, diese also nicht vom Entwickler erst definiert werden müssen.
Der Entwickler kann dadurch einem Modell beliebig Schichten hinzufügen oder entfernen und diese nach seinen Wünschen konfigurieren. 
Die Anzahl an Neuronen innerhalb einer Schicht und die Aktivierungsfunktion der Neuronen in der Schicht werden der Schicht einfach als Parameter übergeben 
und die Aktivierungsfunktion muss ebenfalls nicht selbst definiert werden, denn da reicht es aus, den Namen der Funktion als Parameter anzugeben.
Das Auswählen der Verlustfunktion, des Optimierers und der Metrik verlaufen ebenfalls genauso einfach. Diese werden, nachdem das Modell mit seinen Schichten 
definiert wurde, ganz simpel, wie bei der Auswahl der Aktivierungsfunktion, einer Funktion als Parameter übergeben.
Genau diese Schritte werden in dieser Arbeit durchlaufen, um das \ac{KI}-System zur Bewertung von Anwendungsregeln zu erstellen und anzulernen.

\subsection{Pandas}

Um unter anderem das Importieren der Daten, welche im Praxisprojekt gesammelt wurden, zu ermöglichen, wird die Datenverarbeitungsbibliothek Pandas genutzt.
Pandas ist ein Akronym, welches für panel data steht. Diese Bibliothek basiert dabei auf Tabellen, ähnlich wie bei Excel, und bietet die Möglichkeit
Excel-Dateien, CSV-Dateien und weitere Dateitypen direkt zu importieren oder zu exportieren. Die beiden wichtigsten Datenstrukturen sind dabei die
Series und die Dataframes \cite[vgl. S.253]{NumerischesPython}. 

Series können dabei wie eine zweispaltige Tabelle verstanden werden. Die erste Spalte beinhaltet einen Index, dieser kann
dabei beliebig sein, muss also nicht wie bei einem Array aus Integer-Werten bestehen. Diese Eigenschaft unterscheidet Series von Arrays
und bietet dadurch die Möglichkeit beliebige Indizes zu nutzen und Daten somit als Key-Value-Paare zu speichern. Wird jedoch kein spezieller Index definiert,
so besteht der Index, genau wie bei einem Array, aus aufsteigenden Integer-Werten von 0 bis zur Länge der Series. In der anderen Spalte
werden die eigentlichen Werte gespeichert, diese müssen dabei alle vom selben Datentyp sein \cite[vgl. S.254f.]{NumerischesPython}. 
Weitere Eigenschaften der Datenstruktur Series sind:
\begin{itemize}
    \item Indizierung \cite[vgl. S.256]{NumerischesPython}
    \begin{itemize}
        \item über Index oder Liste von Indizes auf bestimmte Werte eines Series-Objekts zuzugreifen
    \end{itemize}
    \item Filtern nach einer Bedingung, zum Beispiel nur auf Werte zugreifen, die größer als ein Schwellwert sind \cite[vgl. S.256]{NumerischesPython}
    \item Anwendung von mathematischen Funktionen auf gesamtes Series-Objekt möglich \cite[vgl. S.256]{NumerischesPython}
\end{itemize}

Dataframes sind eine weitere Datenstruktur, die die Pandas Bibliothek mitbringt. Ein Dataframe hat, wie ein Series-Objekt, ebenfalls Ähnlichkeiten zu einer Tabelle,
dieses Mal jedoch mit einer unbegrenzten Anzahl an Spalten. Die Werte einer Spalte müssen vom selben Datentyp sein, jedoch können verschiedene Spalten 
auch verschiedene Datentypen besitzen. Da nun mehrere Spalten mit Werten vorhanden sein können, besitzen Dataframes sowohl einen Zeilen- als auch einen
Spaltenindex. Die Indizes sind wieder beliebig wählbar. Zudem können mindestens zwei Series-Objekte zu einem Dataframe
konkateniert werden \cite[vgl. S.263f.]{NumerischesPython}. Also kann über Dataframes gesagt werden, dass sie eine Datenstruktur sind, die aus mehreren einzelnen Series-Objekten bestehen.

Neben den beiden Datenstrukturen liefert die Pandas Bibliothek zahlreiche Funktionen zur Analyse, Bearbeitung und Verwaltung der gespeicherten Daten. Die in dieser Arbeit 
verwendeten Funktionen werden in Kapitel \ref*{chap:Datensatz} an den gesammelten Daten aus dem Praxisprojekt vorgestellt und erläutert.

\subsection{Matplotlib}
Zur Visualisierung von Daten wird in dieser Arbeit die Bibliothek Matplotlib für Python genutzt. Matplotlib bietet die Möglichkeit verschiedenste Diagramme und Darstellungen,
wie zum Beispiel Linien-, Balken-, Tortendiagramme und viele mehr, mit wenig Code zu erstellen. Die erstellten Diagramme können vom Entwickler zudem noch beliebig 
konfiguriert werden \cite[vgl. S.167.]{NumerischesPython}. Das Visualisieren der Daten sorgt für ein besseres Verständnis der Daten im Vergleich zu einer rein textuellen Beschreibung.

Matplotlib eignet sich vor allem in der Verwendung zusammen mit Pandas. Pandas listet Matplotlib als \glqq optionale Abhängigkeit\grqq{}, bedeutet, dass für die Verwendung
von Pandas Matplotlib nicht zwingend benötigt wird, aber es empfohlen wird \cite[vgl. S.253]{NumerischesPython}.
Beide Datenstrukturen der Pandas-Bibliothek besitzen zudem eine Plot-Funktion, also eine Funktion um ein Diagramm aus den Daten zu erstellen, 
die genutzt werden kann, wenn sowohl Pandas als auch Matplotlib als Bibliotheken importiert werden.