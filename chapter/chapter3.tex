\chapter{Algorithmen zum Zeichnen von Bäumen}
\label{chap:kapitel3}

\section{Naiver Algorithmus von Wetherell und Shannon}

Das Paper “Tidy Drawings of Trees” von Charles Wetherell und Alfred Shannon aus dem Jahre 1979, 
welches im IEEE Trans. Softw. Eng. erschienen ist, handelt von verschiedenen Algorithmen zum Zeichnen von Bäumen.
Der erste Algorithmus, der von den beiden Autoren beschrieben und vorgestellt wird, ist ein naiver Algorithmus 
zum Zeichnen von Bäumen. Dieser Algorithmus soll dabei zwei Anforderungen erfüllen. Die erste Anforderung wird dabei
an die Ästhetik des gezeichneten Baumes gestellt. Alle Knoten, die dieselbe Höhe haben, sollen sich auf einer horizontalen
Linie befinden. Jede Höhe hat dabei eine Linie, auf welcher sich die Knoten befinden sollen und diese Linien sollen alle
parallel zueinander sein. Außerdem soll der Algorithmus beim Zeichnen eines Baumes ein physikalisches Limit einhalten.
Das bedeutet, dass der Algorithmus möglichst schmale Bäume zeichnen soll. Jedoch wird die Höhe des Baumes durch diese Anforderungen
nicht eingeschränkt. Stattdessen bestimmt der Baum selbst seine Höhe. \cite[]{q1}

\subsection{Ablauf}
Bevor die Funktionsweise des Algorithmus beschrieben werden kann, muss die Baumstruktur wie folgt definiert sein: Sie benötigt eine Struktur die symbolisch für ein Knoten des Baums steht. Diese Knoten-Struktur muss hierbei ihren Vater kennen, auf ihre Kinder zugreifen können, ihre Position speichern können. Diese Hilfsvariable wird dazu benutzt, um das nächste abzuarbeitende Kind eines Knoten zu identifizieren. 

In Java könnte diese Struktur beispielsweise wie folgt aussehen:
\begin{lstlisting}
class Knoten {
	Knoten vater;
	Knoten[] kinder;
	int hoehe;
	int x, y;
}
\end{lstlisting}
Dieser Algorithmus besitzt zwei Eingabeparameter: Die Wurzel und Höhe des Baumes.
Die Wurzel muss hierbei vom Typ der zuvor definierten Struktur sein. Zu Beginn wird eine Variable definiert:
Ein Array (später Positions-Array genannt), die die jeweils nächst freie X-Position einer Ebene des Baums beinhaltet.
Hiernach wird über die Baumstruktur der Wurzel, in der Pre-Order-Traversierung, traversiert.
Nun werden die X und Y Attribute der Knoten wie folgt bestimmt und gesetzt:

Der derzeitige Knoten bekommt als X-Position den Wert aus dem Positions-Array, in Abhängigkeit von seiner Höhe im Baum.
Hiernach wird die Zahl im Positions-Array inkrementiert. Die Y-Position des Knoten wird nun in Abhängigkeit zur Höhe des Knoten
mit der folgenden Formel berechnet: $$y := 2 * HoeheDesKnotens + 1$$

Dieses Vorgehen wird nun für alle Knoten in dem Baum wiederholt. 

Nach dem Durchlaufen aller Knoten des Baumes sind alle X und Y-Koordinate gesetzt und der Baum kann gezeichnet werden.

\subsection{Vor- und Nachteile}

\section{Verbesserter Algorithmus von Wetherell und Shannon}

\subsection{Ablauf}

\subsection{Vor- und Nachteile}

\section{Algorithmus von Reingold und Tilford}

\subsection{Ablauf}

\subsection{Implementierung in Java}

\subsection{Vor- und Nachteile}

\subsection{Modifizierung des Algorithmus}
