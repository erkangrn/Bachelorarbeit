\label{chap:kapitel3_2}
\section{Verbesserter Algorithmus von Wetherell und Shannon}
Wetherell und Shannon stellen in ihrem Paper einen weiteren, verbesserten Algorithmus zum Zeichnen von Bäumen vor, welcher jedoch
ausschließlich Binärbäume zeichnen kann. Dieser Algorithmus weist die Nachteile des naiven Algorithmus nicht mehr auf.
Dafür definieren sie zwei weitere Anforderungen, die der Algorithmus erfüllen soll.   

\begin{quotation}
	\textit{Aesthetic 2:} In a binary tree, each left son should be positioned
	left of its father and each right son right of its father.\cite[]{q1}
\end{quotation}

In einem Binärbaum hat jeder Knoten maximal ein linkes und maximal ein rechtes Kind. Daher ist es auch logisch, dass jedes linke Kind 
links vom Vater und jedes rechte Kind rechts vom Vater positioniert werden soll. Die zweite weitere Anforderung 

\subsection{Vor- und Nachteile}
