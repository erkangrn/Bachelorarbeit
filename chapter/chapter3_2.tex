\label{chap:kapitel3_2}
\section{Verbesserter Algorithmus von Wetherell und Shannon}
Wetherell und Shannon stellen in ihrem Paper einen weiteren, verbesserten Algorithmus zum Zeichnen von Bäumen vor, welcher jedoch
ausschließlich Binärbäume zeichnen kann. Dieser Algorithmus weist die Nachteile des naiven Algorithmus nicht mehr auf.
Dafür definieren sie zwei weitere Anforderungen, die der Algorithmus erfüllen soll.   

\begin{quotation}
	\textit{Aesthetic 2:} In a binary tree, each left son should be positioned
	left of its father and each right son right of its father.\cite[]{q1}
\end{quotation}

In einem Binärbaum hat jeder Knoten maximal ein linkes und maximal ein rechtes Kind. Daher ist es auch logisch, dass jedes linke Kind 
links vom Vater und jedes rechte Kind rechts vom Vater positioniert werden soll. Zudem soll jeder Vater zentriert über seinen Kindern
stehen. Dieses Verhalten legen Wetherell und Shannon in einer weiteren Anforderung fest.

\begin{quotation}
	\textit{Aesthetic 3:} A parent should be centered over its children.\cite[]{q1}
\end{quotation}

Im Folgenden wird direkt die modifizierte Version des verbesserten Algorithmus betrachtet. Dafür wird die zweite While-Schleife des Programmcodes
mit der Fig. 9 \cite[A modification of Algorithm 3]{q1} ersetzt.

\subsection{Ablauf}

Dieser Algorithmus lässt sich in zwei Phasen unterteilen. In der ersten Phase werden die vorläufigen X-Koordinaten der einzelnen Knoten bestimmt.
In der zweiten Phase werden diese X-Koordinaten bei Bedarf nochmals abgeändert sowie die Y-Koordinate berechnet.

Dieser Algorithmus besitzt zwei Eingabeparameter, nämlich die Wurzel des Baumes und die Höhe des Baumes. Zu Beginn werden zwei
ganzzahlige Arrays, ein Positions-Array und ein Modifikator-Array, definiert. Diese besitzen die Länge ‘Höhe des Baumes’.
Hiernach müssen alle Elemente des Positions-Array mit eins und alle Elemente des Modifikator-Array mit null initialisiert werden.
Zudem wird eine Variable namens ‘ModifikatorSumme’ deklariert und mit null initialisiert.

Nun wird über die Baumstruktur in der Post-Order traversiert. Dabei werden die vorläufigen X-Koordinaten der Knoten wie folgt bestimmt:
Bei der Bestimmung der X-Koordinate wird zwischen vier verschiedenen Fällen unterschieden:
\begin{enumerate}
	\item Der Knoten ist ein Blatt (er besitzt weder ein linken noch rechtes Kind)
	\item Der Knoten besitzt kein linkes Kind
	\item Der Knoten besitzt kein rechtes Kind
	\item Der Knoten besitzt zwei Kinder (links und rechts)
\end{enumerate}

Im ersten Fall entspricht die X-Koordinate dem Eintrag im Positions-Array in Abhängigkeit zu seiner Höhe.
Im zweiten Fall wird der Knoten links, mit einem Offset von eins, von seinem rechten Kind positioniert.
Im dritten Fall wird dieser rechts mit einem Offset von einsvon seinem linken Sohn positioniert. Im letzten Fall wird der Knoten
in der Mitte zwischen seinen Kinder positioniert. Hierbei kann die folgende Formel genutzt werden: $$x = (left.x + right.x) / 2.$$
Hiernach wird der Modifikator bestimmt. Dafür wird der der größere Wert des entweder bereits existierenden Modifikator der Ebene genommen
oder wie folgt berechnet: die nächste freie Position der Ebene subtrahiert mit der zuvor berechneten X-Koordinate.
Nun wird der Wert im Position-Array am Index (Höhe des Knotens im Baum) wie folgt neu berechnet:
X-Koordinate des Knoten plus eins. Zusätzlich wird der spezifische Modifikator des Knotens gesetzt
(Wert aus dem Modifikator-Array am Index: Höhe des Knotens im Baum).

In der zweiten Phase wird zu Beginn eine Variable namens ‘ModifikatorSumme’ deklariert und mit null initialisiert.
Hiernach wird über die Baumstruktur in einer modifizierten Pre-Order nicht rekursiv traversiert. Die ModifikatorSumme entspricht
der Summe aus allen Modifikatoren der Väter eines Knotens. Um diese Summe zu berechnen wird bei dem Besuch eines Kind-Knotens
der spezifische Modifikator des Vaters auf die Summe addiert. Beim Übergang eines Kindes zurück auf den Vater wird entsprechend
der knotenspezifische Modifikator von der globalen ModifikatorSumme abgezogen. Bei der Traversierung wird zwischen
drei Fällen unterschieden: erstmaliger Besuch, Besuch des linken Kindes und Besuch des rechten Kindes. Besitzt ein Knoten den Status
‘erstmaliger Besuch’, so wurde dieser weder besucht noch abgearbeitet. Besitzt ein Knoten den Status ‘Besuch des linken Kindes / rechten Kindes’,
so wurde der linke bzw. rechte Teilbaum fertig abgearbeitet.
Wird ein Knoten erstmalig Besucht, so wird zu ModifikatorSumme der spezifische Knoten-Modifikator addiert. 

Da die Pre-Order Traversierung vorgibt, erst den linken Teilbaum, dann den Knoten und zum Schluss den rechten Teilbaum zu besuchen,
wird zunächst beginnend von der Wurzel aus, der am weitesten links unten stehende Knoten besucht. Abbildung XXX zeigt einen Baum,
welcher mit dem verbesserten Algorithmus gezeichnet wurde. In diesem Beispiel würden wir bei der Wurzel starten und nach Knoten D laufen.
Auf dem Weg dorthin werden die Modifikatoren von A und B auf die globale ModifikatorSumme addiert.
Danach wird die x-Koordinate des Knotens entweder auf die nächste freie x-Koordinate auf der Höhe des Knotens gesetzt oder auf den
Wert der vorläufigen x-Koordinate addiert mit der ModifikatorSumme. Dabei wird der kleinere der beiden Werte genutzt, um sicherzustellen,
dass der Knoten möglichst weit links ist. Wenn der Knoten einen linken Sohn hat und die x-Koordinate des Sohnes größer ist als
die x-Koordinate des Vaters, dann wird die x-Koordinate des Vaters auf die x-Koordinate des Sohnes plus eins gesetzt. 
Damit wird sichergestellt, dass der Knoten nicht direkt über seinem linken Sohn steht, sondern eine Position weiter rechts.
Falls der Knoten nicht die Wurzel ist, der Vater des Knotens bereits besucht und abgearbeitet wurde und die x-Koordinate des Knotens
kleiner als die x-Koordinate des Vaters plus eins ist, dann wird die x-Koordinate des Knotens auf die x-Koordinate des Vaters plus eins gesetzt.
Ähnlich wie im Fall vorher dient diese Überprüfung dazu, den Knoten richtig zu positionieren. Hier wird dadurch verhindert,
dass der Knoten genau unter seinem Vater steht. Stattdessen wird sichergestellt, dass der Knoten rechts von seinem Vater steht,
welcher der rechte Sohn sein muss, da der Vater bereits besucht und positioniert wurde. Nun ist die x-Koordinate des Knotens final
bestimmt worden. Die y-Koordinate wird genau wie im naiven Algorithmus bestimmt, indem die Höhe des Knotens mit zwei multipliziert
und mit einem Offset (hier eins) addiert wird. Zum Schluss wird die nächste freie x-Position auf der Höhe des Knotens bestimmt,
indem die x-Koordinate mit einem Offset (hier zwei) addiert wird. 


\subsection{Implementierung in Java}

\subsection{Vor- und Nachteile}
Durch das Erfüllen der Anforderung, dass jeder Vater über seinen Kindern zentriert werden soll, kann der verbesserte Algorithmus gegen
die Anforderung an das physikalische Limit verstoßen. Dies geschieht, da der Algorithmus die Zentrierung der Väter erzwingt.
Abbildung \todo{Abbildung einfügen} zeigt jenes Verhalten. Daraus schließen die beiden Autoren auf folgendes Theorem:

\begin{quotation}
	\textit{Theorem (Uglification):} Minimum width drawings exist which violate Aesthetic 3 by arbitrary amounts.\cite[]{q1}
\end{quotation}

Bei der schmaleren Variante des Baumes auf Abbildung XXX ist der Vater von Knoten A nicht zentriert über beiden Kindern und verstößt somit 
gegen Aesthetic 3. Die weitere Variante verstößt gegen das physikalische Limit, da der Baum nicht maximal schmal ist. 
Dies stellt hier einen Trade-off zwischen den beiden Anforderungen dar. Es ist somit nicht möglich einen maximal schmalen Baum zu zeichnen,
der auch Aesthetic 3 erfüllt. 

Zudem ist der verbesserte Algorithmus nicht in der Lage beliebige Bäume zu zeichnen, sondern ausschließlich Binärbäume. Jedoch kann er durch
einige geeignete Erweiterungen so modifiziert werden, dass er auch in der Lage wäre beliebige Bäume zu zeichnen.
