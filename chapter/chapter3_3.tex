\label{chap:kapitel3_3}
\section{Algorithmus von Reingold und Tilford}

\label{chap:kapitel3_3_Ablauf}
\subsection{Ablauf}

\subsection{Implementierung in Java}
Um diesen Algorithmus implementieren zu können, muss die zuvor erstellte BinaryKnoten-Klasse um ein 
Attribut erweitert werden: vom Typ Boolean mit dem Namen 'thread'. Zudem wurde eine weitere Klasse 
namens 'Extreme' definiert, die wie zuvor beschrieben, implementiert wurde. Die Extreme-Klasse sieht wie folgt aus:

\begin{figure}[H]
\begin{lstlisting}
private static class Extreme {
    BinaryKnoten knoten;
    int offset;
    int level;
    
    void set(BinaryKnoten k, int offset) {
        this.knoten = k;
        this.level = k.getHoehe();
        this.offset = offset;
    }
}
\end{lstlisting}
    \caption{Implementierung der Extreme-Klasse}
    \label{code:algo3_extreme}
\end{figure}

Ferner wurden die zuvor beschriebenen Prozeduren 'setup' und 'petrify' implementiert. Die implementierte Prozedur
'setup' unterscheidet sich zum zuvor beschriebenen Ablauf. Sie wird nun nicht mehr rekursiv aufgerufen und sie besitzt 
nur ein Eingabeparameter, den Wurzelknoten. Hiernach wird mithilfe der 'traversPostOrder'-Methode aus der 
BinaryKnoten-Klasse über den Baum traversiert.

\begin{figure}[H]
\begin{lstlisting}
public static void setup(BinaryKnoten wurzel) {
    // Initialisierungen von Variablen
    // <...>
    // Ueber den Baum in der Post-Order traversieren
    wurzel.traversPostOrder(k -> {
        BinaryKnoten knoten = (BinaryKnoten) k;

        // Bestimmen der Y-Koordinate
        knoten.setY(2 * knoten.getHoehe() + 1);

        // Vorlaeufige relative X-Koordinate bestimmem
        // <...>
    }
}
\end{lstlisting}
    \caption{Auschnitt aus der setup-Prozedur}
    \label{code:algo3_setup}
\end{figure}

Abweichend zum Ablauf entspricht die Y-Koordinate nicht der Höhe des Knotens. Stattdessen wird diese wie im 
Ablauf aus dem Kapitel \ref{chap:kapitel3_1_Ablauf} berechnet. Dies bietet den Vorteil, 
dass die Methodik zum Zeichnen der Bäume nicht verändert werden muss. Die weitere Implementierung folgt 
der Beschreibung aus dem Ablauf.

Die Implementierung der Prozedur 'petrify', entspricht der Beschreibung aus dem Ablauf.

Hiernach wurde die Prozedur 'algorithmus3' definiert. Diese ruft zu Beginn die beiden Prozeduren, 
'setup' und 'petrify' auf. Hiernach müssen die X-Koordinaten noch angepasst werden, da sich diese 
negativ sein können. Hierfür wird der kleinste X-Wert ermittelt, dessen absoluter Wert 
addiert mit eins in der Variable 'offset' gespeichert wird. Nun werden alle X-Koordinaten des 
Baums mit dem Wert aus 'offset' addiert. 

Zwei Beispielhafte Ergebnisse können in den Abbildungen [] und [] betrachtet werden.


\subsection{Vor- und Nachteile}

\subsection{Modifizierung des Algorithmus}