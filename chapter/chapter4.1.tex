\section{Data Understanding}
\label{chap:DataUnderstanding}

In der Baumansicht auf der linken Seite der Abbildung \ref*{fig:Doors GUI} befindet sich ein Projekt mit dem Namen \glqq RA Application Conditions\grqq{}. In diesem Projekt befinden sich Module, welche
Anwendungsregeln von diversen Komponenten beinhalten. Dafür wurde ein Skript in \acs{DXL} geschrieben, welches über alle Module innerhalb des Projekts iteriert. Wenn das aktuelle Element ein Modul ist,
muss dieses geöffnet werden und falls dieses nicht NULL ist, muss über jedes Objekt in diesem Modul iteriert werden. Bei jedem Objekt muss daraufhin über alle In-Links iteriert werden. Über diese 
In-Links ist es möglich, das Quellmodul zu bestimmen, also das Modul, von dem der Link kommt. Diese Quellmodule sind potentiell relevante Module für das Anlernen des KI-Systems. Nach dem Process Manual für
Anwendungsregeln der Siemens Mobility GmbH \cite[S.32]{q2} muss das Link-Modul des Links den Namen \glqq reference\char95 of\grqq{} tragen. Bei älteren Projekten jedoch wurde als Name 
\glqq 42\char95 reference\char95 of\grqq{} genutzt. Zudem kann der Fall auftreten, dass Quellmodule in einer Sandbox eines Mitarbeiters liegen. Diese Module dürfen nicht berücksichtigt werden, da sie kein Teil
von Projekten der Siemens Mobility GmbH sind. Alle Quellmodule, die den Anforderungen entsprechen, werden in eine Skiplist eingefügt. Skiplisten sind eine Datenstruktur, welche aus einem 
Key-Value-Paar bestehen. 

\begin{lstlisting}[caption={Iterieren über alle Module von RA Application Conditions},captionpos=b, label = lst:searchInLinks]
Folder f = folder("/RA Application Conditions");
void searchInLinks(Folder f){
    for it in f do{
        if (type(it) == "Folder"){
            searchInlinks(folder(it));
        }
        if (type(it) == "Formal"){
            m = read(fullName it, false);
            if (m != null){
                for o in entire m do{
                    for lr in all (o <- "*") do {
                        if (name module lr == "42_reference_of" 
                        || name module lr == "reference_of") {
                            srcModRef = source lr;
                            iOffset = null;
                            iLength = null;
                            if(!(findPlainText(fullName srcModRef, 
                            "Sandbox", iOffset, iLength, false))){
                                put(slModules, fullName srcModRef, 
                                fullName srcModRef);
                                // ...
\end{lstlisting}

Nach dem Iterieren aller Module im Projekt RA Application Conditions, lassen sich zurzeit 115 Module aus 63 Projekten finden, welche einen Link zu Anwendungsregeln besitzen und diese somit importiert haben.
Diese Projekte mit ihren Modulen sind potentielle Kandidaten für einen Datensatz zum Anlernen des KI-Systems. Als nächsten Schritt müssen diese Module genauer betrachtet werden. Es muss geprüft werden,
welche Attribute für das Bewerten von Anwendungsregeln relevant sind. Der Quellcode \ref*{lst:getAttribute} zeigt eine Möglichkeit, wie alle Attribute eines Moduls ausgegeben werden können.
Nach dem Process Manual für Anwendungsregeln müssen Projekte, welche Anwendungsregeln importiert haben, die Attribute REQ Status und REQ Statement beinhalten. Das Attribut REQ Status vom Typ einer
single-enumeration beinhaltet dabei die Bewertung der Anwendungsregel. Die Begründung dafür lässt sich im Attribut REQ Statement in Textform finden\cite[S.14]{q2}. 


\begin{lstlisting}[caption={Alle Attribute eines Moduls ausgeben \cite{q9}}, captionpos=b, label = lst:getAttribute] 
string modAttrName
for modAttrName in attributes (current Module) do
    print modAttrName "\n"
\end{lstlisting}

Die folgende Aufzählung beinhaltet eine Auswahl von Attributen eines beispielhaft gewählten Moduls. 

\begin{multicols}{3}
    \begin{itemize}
        \setlength{\itemsep}{0cm}
        \item Absolute Number
        \item ALinkRef
        \item Compare
        \item Last Modified On
        \item Object Heading
        \item Object Short Text
        \item Object Text
        \item OLE
        \item OLEIconic
        \item Picture
        \item PictureName
        \item PictureNum
        \item REQ Comment\char95 int
        \item REQ Domain
        \item REQ Statement
        \item REQ Status
        \item REQ Type
        \item RTF Pict
        \item STD Explanation
        \item STD Identifier
        \item STD Title
    \end{itemize}
\end{multicols}

Für das Bewerten der Anwendungsregel ist der Object Text, also die Anwendungsregel in Textform, sowie die Attribute REQ Statement und REQ Status relevant, da das KI-System später Vorschläge
für diese beiden Attribute liefern soll. Die Analyse von Modulen von Projekten, 
welche nach 2019 durchgeführt wurden, hat gezeigt, dass dort das Attribut REQ Status den Namen REQ Progress trägt. Darauf muss bei dem Sammeln der Daten und der späteren Aufbereitung dieser geachtet
werden. Zudem wurde dabei deutlich, dass einige Module weder REQ Status bzw. REQ Progress noch REQ Statement als Attribut besitzen. Mögliche Erklärungsansätze dafür wären, dass die Projektausführung
von den Vorgaben des Process Manuals abweichen, dass die Projekte sich noch in einem frühen Stadium befinden und deshalb die Anwendungsregeln nicht bewertet wurden oder dass die Anwendungsregeln lediglich
importiert, aber nicht bewertet wurden.

Außerdem muss beachtet werden, dass ein Projekt mittels eines Copy-Update-Tools die Anwendungsregeln auf eine neue Version updaten kann. Die alten Versionen der Anwendungsregeln werden dann unter ein Objekt
verschoben, welches die Überschrift \glqq Old Version Objects, to be reviewed and deleted: \grqq{} hat. Diese Anwendungsregeln dürfen nicht verwendet werden, da diese veraltet sein könnten. 