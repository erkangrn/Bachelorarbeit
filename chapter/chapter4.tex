\chapter{Konklusion}
\label{chap:kapitel4}
\erstelltvon{Garan/Treulieb}
Es wurden drei verschiedene Algorithmen zum Zeichnen von Bäumen im Ebenen-Layout vorgestellt, erklärt und in Java implemeniert.
Die Abbildungen \ref{pic:komplex_a1}, \ref{pic:komplex_a2} (Algorithmus von Wetherhell und Shannon ohne Modifizierung), 
\ref{pic:komplex_a2v}, \ref{pic:komplex_a3} zeigen allesamt den gleichen Baum, jedoch gezeichnet mit den verschiedenen Algorithmen. 

Der erste Algorithmus liefert den schmalsten Baum im Vergleich zu den anderen Algorithmen. Dafür werden die Knoten aber maximal weit nach links
platziert, was dafür sorgt, dass der Baum unübersichtlich wirkt und dass die Beziehung zwischen linkem und rechten Kind verloren geht. Wird der
Knoten C von der Abbildung \ref{pic:komplex_a1} und seine Kinder betrachtet, dann wirkt es so, als hätte C zwei rechte Kinder.

Beim zweiten Algorithmus ohne die Modifikation wird ein ästhetisch ansprechender Baum erzeugt. Jedoch ist der Baum, wie Abbildung
\ref{pic:komplex_a2} zeigt, nicht maximal schmal. Außerdem produziert dieser Algorithmus nicht in jedem Fall Spiegelbilder, wie in der
Abbildung \ref{pic:WS_Spiegel} deutlich wird. 

Wird der zweite Algorithmus mit der Modifizierung versehen, wird ein unübersichtlicher Baum gezeichnet. Bei der Abbildung \ref{pic:komplex_a2v}
wird dies deutlich. Die Väter sind nicht über den Kindern zentriert, was bei der Beziehung A-B-C erkannt werden kann. Zudem werden lange Kanten
gezeichnet, was bei der Kante zwischen den Knoten J und N deutlich wird.

Im Vergleich zu den anderen Algorithmen produziert der Algorithmus von Reingold und Tilford den ästhetisch ansprechendsten Baum, da dieser
alle vier Anforderungen an die Ästhetik erfüllt. Deutlich wird das beim Betrachten der Abbildungen \ref{pic:komplex_a3} und \ref{pic:TR_Spiegel}.
Dies kommt auf Kosten des physikalischen Limits, da der Baum nicht maximal schmal ist. Jedoch ist gerade in der heutigen Zeit das Einhalten
eines physikalischen Limits nicht mehr so wichtig wie in den 1980er Jahren. 

