\chapter{Schluss}
\label{chap:Schluss}
Das Ziel dieses Praxisprojekts war es, einen für das Anlernen eines KI-Systems geeigneten Datenpool zu erstellen. Nach dem Model \acs{CRISP-DM} wurden dabei die Projekte, welche
Anwendungsregeln genutzt haben, untersucht und analysiert. Zu Beginn wurde zunächst nach potentiell relevanten Projekten in der \acs{DOORS}-Datenbank gesucht. Realisiert wurde diese Suche mit Skripten
in der Sprache \acs{DXL}.
Diese Projekte und ihre dazugehörigen Module wurden daraufhin genauer untersucht. Dabei wurde geprüft, welche Attribute relevant sein könnten und welche nicht. Das Process Manual wurde dabei genutzt,
um ein besseres Verständnis der einzelnen Attribute zu erlangen und die vorgeschriebene Vorgehensweise beim Bewerten von Anwendungsregeln nachzuvollziehen. An dieser Stelle fiel auf, dass einige
Projekte dabei ihre importierten Anwendungsregeln nicht oder noch nicht bewertet hatten. Deshalb mussten jene Projekte, welche keine Bewertungen zu ihren Anwendungsregeln besaßen, aussortiert werden,
da diese keinen Mehrwert für den zu erstellenden Datenpool boten. Im nächsten Schritt mussten neue Module für jedes Projekt erstellt werden. Die neu erstellten Module durften dabei nur für das Anlernen
des KI-Systems relevante Attribute besitzen. Mithilfe dieser neu erstellten Module für jedes Projekt konnten dann alle Anwendungsregeln in den Modulen mit ihren Bewertungen in eine .csv-Datei geschrieben 
werden. Abschließend wurden die Module jedes Projekts aufgeteilt. Dafür wurde für jede Komponente oder jedes System, dessen Anwendungsregeln in einem Projekt bewertet wurden, ein eigenes Modul erstellt. 
Abschließend wurden diese aufgeteilten Module dann in ihren jeweiligen Project Answers Ordner verschoben, wie es das Process Manual für Anwendungsregeln vorgibt \cite[S.32]{q2}. 

Durch dieses Praxisprojekt wurde ein Datenpool über 195.518 bewertete Anwendungsregeln erstellt, welcher für das Anlernen eines unterstützenden KI-Systems zum Bewerten von Anwendungsregeln genutzt werden
kann. Dieser Bericht betrachtet dabei das Erstellen eines KI-Systems nicht und bietet deshalb die Möglichkeit, als Grundlage für eine weitere Arbeit zu dienen, die sich mit der Erstellung und dem Anlernen
eines KI-Systems beschäftigt.
