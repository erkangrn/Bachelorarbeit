\chapter{Fazit}
\label{chap:Fazit}
Das Ziel dieser Bachelorarbeit war der Entwurf eines \ac{KI}-Modells, das Vorschläge zur Bewertung von Anwendungsregeln machen sollte. Dieses Modell wurde
in der Programmiersprachen Python mit der \ac{DL}-Bibliothek Keras erstellt. Es wurde gezeigt, wie ein Modell sowohl mit der \glqq Sequential\grqq{}-Klasse
als auch mit der funktionalen \ac{API} erstellt werden kann. Zudem wurde gezeigt, wie ein Modell trainiert wird und welche Möglichkeiten zur Verbesserung eines 
Modells bestehen. Durch Visualisierungen 
\section{Ausblick}
\label{chap:Ausblick}
Im Anschluss an diese Bachelorarbeit kann das erstellte Modell genutzt werden, um eine Erweiterung für das Anforderungsmanagement-Tool \ac{DOORS} zu erstellen
und somit direkt in \ac{DOORS} integriert zu werden.
Dabei kann statt der Ausgabe auf der Konsole eine grafische Benutzeroberfläche erstellt und verwendet werden, um die Vorschläge des Modells ansehnlicher auszugeben.
Diese Benutzeroberfläche könnte dann beispielsweise die Möglichkeit besitzen, einen Vorschlag direkt in das Modul in \ac{DOORS} zu importieren. 
Zudem könnte die Ausgabe auch mehrere Statements, sortiert nach prognostizierter Wahrscheinlichkeit, beinhalten, die der Benutzer dann durchgehen kann, um so einen Überblick 
zu erhalten, weshalb ein Projekt eine Anwendungsregel auf eine spezifische Art bewertet hat. Auch könnten die erstellten Visualisierungen Teil der Erweiterung in \ac{DOORS} sein.
Zum Beispiel könnte ein Benutzer sich ein Tortendiagramm anzeigen lassen, welches zeigt, wie oft eine Anwendungsregel mit einer bestimmten Ausprägung bewertet wurde.
\\ \\
Es ist zudem damit zu rechnen, dass mit zunehmender Zeit, der Trainingsdatensatz größer wird, da laufend Projekte Anwendungsregeln bewerten. Bei einem größeren Trainingsdatensatz 
kann auch von einer besseren Leistung des Modells ausgegangen werden. Außerdem könnte, statt der Prognose des Statements als Klasse, Natural Language Processing (NLP) genutzt werden, 
um für das Statement einen Text zu generieren. 
\\ \\
Ein weiterer Aspekt ist das Ausführen der \ac{KI} in einer Cloud. Das Anlernen eines \ac{KI}-Modells
kann bei zunehmender Komplexität und Größe des Datensatzes zeitintensiv werden. Das Ausführen in einer Cloud würde diesen Prozess beschleunigen,
da diese Server über eine leistungsstärkere Hardware verfügen, als der Rechner, auf dem diese Arbeit geschrieben und erstellt wurde. Der Teil wurde in der Arbeit aufgrund von 
Kosten und möglichen Sicherheitsbedenken bei der Datensicherheit nicht erwähnt, sollte aber in Zukunft beachtet und evaluiert werden. 