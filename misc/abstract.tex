%\chapter*{Überblick}
\section*{Kurzfassung}
Anforderungen sind bestimmte Bedingungen oder Fähigkeiten, welche ein System erfüllen oder besitzen muss. Das korrekte Erfüllen von Anforderungen ist essenziell für den Erfolg von Projekten.
Eine besondere Form von Anforderungen stellen die Anwendungsregeln dar. Um einen sicheren und zuverlässigen Einsatz von Komponenten in einem Projekt zu gewährleisten,
ist eine Erfüllung der Anwendungsregeln der im Projekt genutzten Komponenten vorausgesetzt. Diese Bachelorarbeit wird sich mit der Erstellung eines unterstützenden KI-Systems beschäftigen,
das in der Lage sein soll, mögliche Vorschläge für Bewertungen zu Anwendungsregeln zu liefern. Dafür wird ein Deep Learning Modell mit Daten über Projekte und 
ihren Anwendungsregeln der Siemens Mobility GmbH trainiert. Diese Daten müssen vorher in eine für das Anlernen eines Modells geeignete Form 
gebracht werden. Diese Schritte der Datenvorverarbeitung werden hier vorgestellt. Zudem werden in dieser Arbeit Kenntnisse darüber vermittelt, 
wie künstliche Intelligenz funktioniert, was der Unterschied zwischen KI, maschinellem Lernen und Deep Learning ist und wie solch ein Modell aufgebaut, angewendet und getestet werden kann. 

\vfill\vfill\vfill\vfill\vfill\vfill
\section*{Abstract}
Requirements are certain conditions that a system must fulfill or capabilities that it must possess. The correct fulfillment of requirements is essential for the success of projects.
Application rules are a special type of requirements. To ensure the safe and reliable use of components in a project,
a fulfillment of the application rules from the components used in the project is mandatory. This bachelor thesis will deal with the creation of a supporting AI system,
that should be able to predict possible suggestions for the evaluation of application rules. For this purpose, a deep learning model will be trained with data on projects and 
their application rules from the Siemens Mobility GmbH. This data must first be transformed into a form suitable for the training of an AI model. 
The required data preprocessing steps are presented here. In addition, this thesis will provide knowledge about, 
how artificial intelligence works, what the difference between AI, machine learning and deep learning is and how such a model can be built, applied and tested.
\vfill\vfill\vfill\vfill\vfill\vfill