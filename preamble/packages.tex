% **************************** PACKAGE SETUP *******************************
\usepackage[ngerman]{babel}          % Lokalisierung von Typographie, Silbentrennung, etc.

\usepackage{ucs}                     % Erweiterte Unterstützung von UTF-8-Kodierung
\usepackage[utf8x]{inputenc}         % Unterstützung von UTF-8 in Eingabe-Dateien
\usepackage[T1]{fontenc}             % Zeichensatzkodierung von LaTeX (Cork-Kodierung)
\usepackage{helvet,courier,mathptmx} % Verwendete Schriftarten

\usepackage[headsepline, plainheadsepline, plainfootsepline] {scrlayer-scrpage}

\usepackage{amsmath}                 % Mathematische Infrastruktur für LaTeX der AMS
\usepackage{amsfonts}                % Mathematische Schriftarten
\usepackage{amssymb}                 % Mathematische Symbole
\usepackage{amsthm}                  % Erweiterung der Theorem-Umgebungen
\usepackage[]{units}					 % für \unit-Befehl
%\usepackage[amssymb]{siunits}		 % für SI-Einheiten (amssymb definiert den Befehl \square des amssymb Packages um. Ist dies nicht gewünscht kann die Option squaren verwendet werden. Dann muss für die SI-Einheiten \squaren anstatt \square verwendet werden.)

%\usepackage{fancyhdr}                % Erweiterte Konfiguration von Kopf/Fußzeile
\usepackage{hyperref}                % Querverweise, Hyperlink, pdf-Konfiguration, etc.

\usepackage{float}                   % Selbstdefinierte Floating-Umbgebungen
\usepackage{tabularx}                % Tabellen mit einstellbarer Spaltenbreite
\usepackage[labelfont=bf]{caption}   % Anpassen der Abbildungs- und Tabellenbeschriftungen

\usepackage{algpseudocode}           % Algorithmen als Pseudocode (basiert auf algorithmicx)
\usepackage{listings}                % Quellcode-Satz (z.B. mit Syntax-Hervorhebung)

\usepackage[pdftex,
%draft							%Figures werden nur als Platzhalter eingeblendet
]{graphicx}						% Erweiterte Unterstützung von Graphiken
\usepackage{textpos}                 % Beliebig platzierte Textboxen
\usepackage{xcolor}                  % TeX-Engine-unabhängige Definition von Farben
\usepackage[printonlyused]{acronym}


\usepackage{pdfpages}

\usepackage[numbers]{natbib}         % Weiter Optionen für die Bibliographie
\usepackage{todonotes}
\usepackage{tabulary}
\usepackage{IEEEtrantools}

\usepackage{graphicx}

